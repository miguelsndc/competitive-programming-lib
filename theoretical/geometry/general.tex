\section*{Geometry Identities and Transformations (Competitive Programming)}

\subsection*{Basic Identities and Vector Operations}

\subsubsection*{2D Geometry}
\begin{itemize}
    \item \textbf{Point Representation}: $P=(x, y)$
    \item \textbf{Vector from $P_1$ to $P_2$}: $\vec{P_1P_2} = (x_2-x_1, y_2-y_1)$
    \item \textbf{Distance between two points} $P_1(x_1, y_1)$ and $P_2(x_2, y_2)$:
    $d = \sqrt{(x_2 - x_1)^2 + (y_2 - y_1)^2}$
    \item \textbf{Midpoint of a segment} connecting $P_1(x_1, y_1)$ and $P_2(x_2, y_2)$:
    $M = \left(\frac{x_1 + x_2}{2}, \frac{y_1 + y_2}{2}\right)$
    \item \textbf{Area of a triangle} with vertices $(x_1, y_1), (x_2, y_2), (x_3, y_3)$:
    $A = \frac{1}{2} |x_1(y_2 - y_3) + x_2(y_3 - y_1) + x_3(y_1 - y_2)|$
    \item \textbf{Signed Area of Triangle}: $\frac{1}{2}((x_2-x_1)(y_3-y_1) - (x_3-x_1)(y_2-y_1))$ (positive for CCW, negative for CW)
    \item \textbf{Cross Product (2D)} for vectors $\vec{A}=(x_1, y_1)$ and $\vec{B}=(x_2, y_2)$:
    $\vec{A} \times \vec{B} = x_1 y_2 - x_2 y_1$ (scalar value, positive if $\vec{B}$ is CCW from $\vec{A}$)
    \item \textbf{Dot Product (2D)} for vectors $\vec{A}=(x_1, y_1)$ and $\vec{B}=(x_2, y_2)$:
    $\vec{A} \cdot \vec{B} = x_1 x_2 + y_1 y_2 = |\vec{A}||\vec{B}|\cos\theta$
\end{itemize}

\subsubsection*{3D Geometry}
\begin{itemize}
    \item \textbf{Point Representation}: $P=(x, y, z)$
    \item \textbf{Vector from $P_1$ to $P_2$}: $\vec{P_1P_2} = (x_2-x_1, y_2-y_1, z_2-z_1)$
    \item \textbf{Distance between two points} $P_1(x_1, y_1, z_1)$ and $P_2(x_2, y_2, z_2)$:
    $d = \sqrt{(x_2 - x_1)^2 + (y_2 - y_1)^2 + (z_2 - z_1)^2}$
    \item \textbf{Dot Product (3D)} for vectors $\vec{A}=(x_1, y_1, z_1)$ and $\vec{B}=(x_2, y_2, z_2)$:
    $\vec{A} \cdot \vec{B} = x_1 x_2 + y_1 y_2 + z_1 z_2 = |\vec{A}||\vec{B}|\cos\theta$
    \item \textbf{Cross Product (3D)} for vectors $\vec{A}=(x_1, y_1, z_1)$ and $\vec{B}=(x_2, y_2, z_2)$:
    $\vec{A} \times \vec{B} = (y_1 z_2 - y_2 z_1, x_2 z_1 - x_1 z_2, x_1 y_2 - x_2 y_1)$
    \item \textbf{Volume of Tetrahedron} with vertices $P_0, P_1, P_2, P_3$:
    $\frac{1}{6} |\det(\vec{P_0P_1}, \vec{P_0P_2}, \vec{P_0P_3})|$ (scalar triple product)
\end{itemize}

\subsection*{Lines and Segments (2D)}

\begin{itemize}
    \item \textbf{Line Equation}: $Ax + By + C = 0$ (normal vector $(A, B)$)
    \item \textbf{Slope}: $m = (y_2-y_1)/(x_2-x_1)$
    \item \textbf{Perpendicular Slope}: $-1/m$
    \item \textbf{Point-Slope Form}: $y - y_1 = m(x - x_1)$
    \item \textbf{Intersection of two lines} ($A_1x+B_1y+C_1=0$, $A_2x+B_2y+C_2=0$):
    $x = \frac{B_1C_2 - B_2C_1}{A_1B_2 - A_2B_1}$, $y = \frac{A_2C_1 - A_1C_2}{A_1B_2 - A_2B_1}$ (check for $A_1B_2 - A_2B_1 = 0$ for parallel/coincident lines)
    \item \textbf{Distance from a point $P_0(x_0, y_0)$ to line $Ax+By+C=0$}:
    $d = \frac{|Ax_0 + By_0 + C|}{\sqrt{A^2 + B^2}}$
    \item \textbf{Orientation of three points} $P_1, P_2, P_3$: Use cross product $\vec{P_1P_2} \times \vec{P_1P_3}$.
        \begin{itemize}
            \item $> 0$: Counter-clockwise (left turn)
            \item $< 0$: Clockwise (right turn)
            \item $= 0$: Collinear
        \end{itemize}
    \item \textbf{Segment Intersection}: Check orientation of $(P_1, P_2, P_3)$, $(P_1, P_2, P_4)$, $(P_3, P_4, P_1)$, $(P_3, P_4, P_2)$. Special handling for collinear segments.
\end{itemize}

\subsection*{Circles (2D)}
\begin{itemize}
    \item \textbf{Equation}: $(x-h)^2 + (y-k)^2 = r^2$, center $(h,k)$, radius $r$.
    \item \textbf{Area}: $A = \pi r^2$
    \item \textbf{Circumference}: $C = 2\pi r$
    \item \textbf{Distance between two circle centers}: $d = \sqrt{(h_2-h_1)^2 + (k_2-k_1)^2}$
    \item \textbf{Intersection of two circles}:
        \begin{itemize}
            \item $d > r_1 + r_2$: No intersection (disjoint)
            \item $d = r_1 + r_2$: One intersection point (external tangent)
            \item $|r_1 - r_2| < d < r_1 + r_2$: Two intersection points
            \item $d = |r_1 - r_2|$: One intersection point (internal tangent)
            \item $d < |r_1 - r_2|$: No intersection (one inside other)
            \item $d=0$ and $r_1=r_2$: Coincident circles (infinite points)
        \end{itemize}
\end{itemize}

\subsection*{Polygon Properties (2D)}
\begin{itemize}
    \item \textbf{Area of a simple polygon} with vertices $(x_0, y_0), \dots, (x_{n-1}, y_{n-1})$ (Shoelace Formula):
    $A = \frac{1}{2} \left| \sum_{i=0}^{n-1} (x_i y_{i+1} - x_{i+1} y_i) \right|$, where $(x_n, y_n) = (x_0, y_0)$.
    (Sum of cross products: $\sum_{i=0}^{n-1} \vec{OP_i} \times \vec{OP_{i+1}}$ where $O$ is origin)
    \item \textbf{Convex Polygon}: All internal angles are less than or equal to $180^\circ$.
    \item \textbf{Concave Polygon}: Has at least one internal angle greater than $180^\circ$.
    \item \textbf{Point in Polygon Test (Ray Casting)}: Draw a ray from the point in any direction (e.g., positive x-axis) and count intersections with polygon edges. If odd, point is inside; if even, outside. Handle horizontal edges and vertices carefully.
    \item \textbf{Point in Convex Polygon Test}: Check if the point is consistently on one side (e.g., left) of all directed edges of the polygon.
\end{itemize}

\subsection*{Convex Hull (2D)}
\begin{itemize}
    \item The smallest convex polygon enclosing a given set of points.
    \item Algorithms: Graham Scan ($O(N \log N)$), Monotone Chain ($O(N \log N)$).
    \item \textbf{Graham Scan steps}:
        \begin{enumerate}
            \item Find lowest-most (and leftmost if ties) point $P_0$.
            \item Sort all other points by angle with $P_0$ (or cross product $\vec{P_0P_i} \times \vec{P_0P_j}$).
            \item Iterate through sorted points, maintaining a stack. If new point creates a right turn with top two stack points, pop from stack until left turn.
        \end{enumerate}
\end{itemize}

\subsection*{Sweep Line Algorithms}
\begin{itemize}
    \item Used for problems involving geometric objects (segments, rectangles) where a vertical (or horizontal) line sweeps across the plane.
    \item Often involves a `std::set` or segment tree to maintain active objects on the sweep line.
\end{itemize}

\subsection*{Floating Point Precision Issues}
\begin{itemize}
    \item Use a small epsilon ($\varepsilon \approx 10^{-9}$ to $10^{-12}$) for comparisons instead of direct equality:
        \begin{itemize}
            \item $a == b \Rightarrow |a-b| < \varepsilon$
            \item $a < b \Rightarrow a < b - \varepsilon$
            \item $a > b \Rightarrow a > b + \varepsilon$
        \end{itemize}
    \item Prefer integer arithmetic where possible (e.g., use cross products for collinearity/orientation instead of slopes).
\end{itemize}

\subsection*{Geometric Transformations (2D Homogeneous Coordinates)}
A point $(x, y)$ is represented as a column vector $\begin{pmatrix} x \\ y \\ 1 \end{pmatrix}$.

\subsubsection*{Translation by $(t_x, t_y)$}
$T = \begin{pmatrix} 1 & 0 & t_x \\ 0 & 1 & t_y \\ 0 & 0 & 1 \end{pmatrix}$

\subsubsection*{Scaling by $(s_x, s_y)$ relative to origin}
$S = \begin{pmatrix} s_x & 0 & 0 \\ 0 & s_y & 0 \\ 0 & 0 & 1 \end{pmatrix}$

\subsubsection*{Rotation by angle $\theta$ relative to origin (counter-clockwise)}
$R = \begin{pmatrix} \cos\theta & -\sin\theta & 0 \\ \sin\theta & \cos\theta & 0 \\ 0 & 0 & 1 \end{pmatrix}$

\subsubsection*{Shear}
\begin{itemize}
    \item X-Shear (parallel to x-axis, factor $k_x$):
    $Sh_x = \begin{pmatrix} 1 & k_x & 0 \\ 0 & 1 & 0 \\ 0 & 0 & 1 \end{pmatrix}$
    \item Y-Shear (parallel to y-axis, factor $k_y$):
    $Sh_y = \begin{pmatrix} 1 & 0 & 0 \\ k_y & 1 & 0 \\ 0 & 0 & 1 \end{pmatrix}$
\end{itemize}

\subsubsection*{Reflection}
\begin{itemize}
    \item Across X-axis:
    $Ref_x = \begin{pmatrix} 1 & 0 & 0 \\ 0 & -1 & 0 \\ 0 & 0 & 1 \end{pmatrix}$
    \item Across Y-axis:
    $Ref_y = \begin{pmatrix} -1 & 0 & 0 \\ 0 & 1 & 0 \\ 0 & 0 & 1 \end{pmatrix}$
    \item Across origin:
    $Ref_O = \begin{pmatrix} -1 & 0 & 0 \\ 0 & -1 & 0 \\ 0 & 0 & 1 \end{pmatrix}$
    \item Across line $y=x$:
    $Ref_{y=x} = \begin{pmatrix} 0 & 1 & 0 \\ 1 & 0 & 0 \\ 0 & 0 & 1 \end{pmatrix}$
\end{itemize}

\subsection*{Geometric Transformations (3D Homogeneous Coordinates)}
A point $(x, y, z)$ is represented as a column vector $\begin{pmatrix} x \\ y \\ z \\ 1 \end{pmatrix}$.

\subsubsection*{Translation by $(t_x, t_y, t_z)$}
$T = \begin{pmatrix} 1 & 0 & 0 & t_x \\ 0 & 1 & 0 & t_y \\ 0 & 0 & 1 & t_z \\ 0 & 0 & 0 & 1 \end{pmatrix}$

\subsubsection*{Scaling by $(s_x, s_y, s_z)$ relative to origin}
$S = \begin{pmatrix} s_x & 0 & 0 & 0 \\ 0 & s_y & 0 & 0 \\ 0 & 0 & s_z & 0 \\ 0 & 0 & 0 & 1 \end{pmatrix}$

\subsubsection*{Rotation by angle $\theta$ (counter-clockwise)}
\begin{itemize}
    \item About X-axis:
    $R_x(\theta) = \begin{pmatrix} 1 & 0 & 0 & 0 \\ 0 & \cos\theta & -\sin\theta & 0 \\ 0 & \sin\theta & \cos\theta & 0 \\ 0 & 0 & 0 & 1 \end{pmatrix}$
    \item About Y-axis:
    $R_y(\theta) = \begin{pmatrix} \cos\theta & 0 & \sin\theta & 0 \\ 0 & 1 & 0 & 0 \\ -\sin\theta & 0 & \cos\theta & 0 \\ 0 & 0 & 0 & 1 \end{pmatrix}$
    \item About Z-axis:
    $R_z(\theta) = \begin{pmatrix} \cos\theta & -\sin\theta & 0 & 0 \\ \sin\theta & \cos\theta & 0 & 0 \\ 0 & 0 & 1 & 0 \\ 0 & 0 & 0 & 1 \end{pmatrix}$
\end{itemize}
